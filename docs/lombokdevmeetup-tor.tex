\documentclass{article}

\begin{document}

\title{Kerangka Acuan Kerja \protect\\2nd Lombok Dev Meetup}
\maketitle
\section{Pengantar}

Teknologi informasi adalah salah satu dari bidang pengetahuan manusia yang perkembangannya paling cepat. Bahkan di dunia kerja sendiri, banyak pengetahuan yang berputar terlalu cepat untuk ditangkap oleh seorang individu. Internet dan semangat berkomunitas dapat membantu individu untuk tetap memperoleh pengetahuan baru dan tidak tertinggal dari kemajuan teknologi. Komunitas juga mengizinkan individu untuk saling berbagi dan berjejaring.

Di kota-kota besar, komunitas-komunitas seperti ini sudah sangat banyak dan berdiri dengan kuat. Komunitas-komunitas tersebut rutin mengadakan pertemuan langsung/sesi berbagi secara luring dan gratis sebagai solusi berbagi yang paling mudah dan paling banyak nilainya dari sisi manfaat. Namun di kota kecil seperti Mataram, pertemuan-pertemuan komunitas belum begitu menonjol dan masih bergerak secara sporadis.

Sebagai tanggapan atas kondisi ini, Lombok Dev Meetup dibentuk. Lombok Dev Meetup pertama kali diinisiasi pada tahun 2016 dan meetup pertama berhasil terselenggara pada bulan November 2016.

\section{Tujuan Kegiatan}

\textit{Meetup} ini diadakan untuk :

\begin{itemize}
  \item Menginisiasi semangat berkomunitas dan berkolaborasi dari para peserta (lihat definisi bagian Peserta)
  \item Mendorong para peserta untuk saling berbagi tentang hal-hal baru yang mereka pelajari baik dalam akademik maupun pekerjaan
  \item Menjadi wadah untuk pertemuan sesama peserta dan profesi terkait
\end{itemize}

\section{Hasil yang Diharapkan}
\begin{itemize}
  \item Peningkatan SDM secara berkelanjutan. Dengan komunitas yang kuat, asas saling berbagi dan kecenderungan akan hal baru, peserta memperoleh pengetahuan-pengetahuan baru yang berguna dari peserta \textit{meetup} lainnya. Sehingga, SDM di daerah dapat mengejar perolehan pengetahuan yang sama dengan SDM di kota besar.
  \item Generasi yang lebih muda menjadi lebih siap untuk menghadapi dinamika kemajuan teknologi yang cepat. 
  \item Masyarakat terinspirasi untuk mengadakan acara / mengulang kembali acara serupa dengan output yang lebih baik dan matang di masa yang akan datang.
  \item Entitas bisnis dan pencari kerja dapat saling bertemu dan berjejaring, secara tidak langsung membantu meningkatkan perekonomian daerah setempat.
  \item Peserta mendapat kesempatan untuk menyalurkan dan mengembangkan kemampuan \textit{public speaking}-nya.
\end{itemize}

\section{Peserta}

Peserta adalah (namun tidak terbatas) berasal dari :

\begin{itemize}
  \item Pelajar
  \item Mahasiswa
  \item Developer
  \item System/Network Engineer
  \item Pemrogram
  \item Sysadmin
  \item Profesi terkait lainnya di dunia IT
\end{itemize}

\section{Pemateri}

Pemateri berasal dari peserta yang ingin berbagi pengetahuan, inspirasi dan cerita. Pada \textit{Meetup} kedua ini, pematerinya (dan materi yang dibawakan) antara lain :
\begin{itemize}
  \item Rizky Ariestiyansyah - "Ngoprek Realitas Maya di Web (Web Virtual Reality)"
  \item Eby Sofyan - "Android Devs, 5 Menit Mesra Bersama Kotlin"
  \item Herpiko Dwi Aguno - "Revolusi Kultur DevOps"
  \item Hayi Nukman - "Jenkins, Perkakas CI/DI untuk DevOps"
  \item Deni Marswandi - "Membangun Rest API dengan Golang"
  \item Lalu Erfandi Maulana Yusnu - "Tensorflow dan Tebak Gambar Bersama Komputer"
  \item Hadian Wijaya - "Berajah Pinak Web kadu Base Sasak"
  \item Kominfo Ditjen Keamanan Teknologi Informasi (sedang dalam konfirmasi) - "Infrastruktur Kunci Publik"
  \item Dr. Sapto Sutardi - "Bagaimana Tidur Teratur Sebagai Pemrogram"
  \item Nino - "Pengenalan Swift"
\end{itemize}

\section{Kegiatan}

Kegiatan akan dilaksanakan pada 15 Oktober 2017 di ITEC, Mataram. Secara garis besar, selain berkumpul kegiatan utamanya adalah sesi \textit{tech talks}, sebagian dikemas dalam bentuk \textit{lightning talk} (5-10 menit) agar acara ini dapat menampung berbagai topik dari sekian banyak fragmentasi di dunia IT. Keberlangsungan acara dan penetapan waktu diatur oleh seorang Moderator. Adapun susunan acara adalah sebagai berikut :

\begin{itemize}
  \item 09.00 - 09.30 : Registrasi
  \item 09.30 - 10.00 : Pembukaan dan sambutan dari sponsor utama
  \item 10.00 - 10.20 : Materi ke-1 (20 menit)
  \item 10.20 - 10.30 : Tanya jawab
  \item 10.30 - 10.40 : Materi ke-2 
  \item 10.40 - 10.50 : Tanya jawab
  \item 10.50 - 11.00 : Materi ke-3
  \item 11.00 - 11.10 : Tanya jawab
  \item 11.10 - 11.20 : Materi ke-4
  \item 11.20 - 11.30 : Tanya jawab
  \item 11.30 - 11.50 : Materi ke-5 (20 menit)
  \item 11.50 - 12.00 : Tanya jawab
  \item Ishoma sampai pukul 13.00
  \item 13.00 - 13.20 : Materi ke-6 (20 menit)
  \item 13.20 - 13.30 : Tanya jawab
  \item 13.30 - 13.40 : Materi ke-7
  \item 13.40 - 13.50 : Tanya jawab
  \item 13.50 - 14.00 : Materi ke-8
  \item 14.00 - 14.10 : Tanya jawab
  \item 14.10 - 14.20 : Materi ke-9
  \item 14.20 - 14.30 : Tanya jawab
  \item 14.30 - 14.40 : Materi ke-10
  \item 14.40 - 14.50 : Tanya jawab
  \item Ishoma sampai pukul 16.00
  \item 16.00 - selesai : Sesi berbaur, diskusi bebas dan penutupan
\end{itemize}

\end{document}